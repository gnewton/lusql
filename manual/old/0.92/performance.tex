
\subsection{Tuning}
\subsubsection{Java VM options}
\begin{mlist}
\item {\tt -XX:+AggressiveOpts} was found to increase thoughput on this configuration.
\end{mlist}

\subsubsection{Lucene indexing}
Using large RAM buffer makes indexing faster. 
However, with very large RAM buffer size, the delay in flushing the buffer can
be longer than the JDBC connection is comfortable with, and can cause a
timeout. 
This can be solved by setting rather small RAM buffer sizes, like the Lucene
default value of 16MB.

\subsubsection{SQL Database and queries}
\begin{mlist}
\item Make sure fields that are used in {\tt WHERE} clauses are appropriately indexed.
\end{mlist}

\paragraph{MySQL}
\begin{mlist}

\item Add {\tt cacheResultSetMetadata=true}\footnote{\url{http://dev.mysql.com/doc/refman/5.0/en/connector-j-reference-configuration-properties.html}} 
  to the connection String for MySql. 
\end{mlist}


\subsection{Indexing Performance}
Here is the indexing performance for examples 2--4 of the tutorial in
Section \ref{tutorial}, indexing all of the records (6,409,484) from each
query.  

\subsubsection{Example 2}
Total indexing time: {\bf 5675s (1h 34m 35s)} \\
Optimizing time: {\bf 2725s (m s)} \\
Index size: {\bf 21GB} \\

\noindent LuSql command:
\begin{lstlisting}[backgroundcolor=\color{grey},language=Bash]
java -XX:+AggressiveOpts -Xms1000m -Xmx1000m\
 -jar lusql.jar  -q "select Article.id as id,\
 Article.title as ti, Article.abstract as ab,\
 Publisher.name as pub, Journal.title as jo,\
 Journal.issn,Volume.number as vol,\
 Volume.coverYear as year, Issue.number as iss,\
 Article.doi as doi,Article.title as title,\
 Article.abstract as abstract, Article.startPage as startPage,\
 Article.endPage as endPage from Publisher, Journal,\
 Volume, Issue, Article where Publisher.id=Journal.publisherId\
 and Journal.id = Volume.journalId and Volume.id\
 =Issue.volumeId and Issue.id = Article.issueId " -c\
 "jdbc:mysql://dbhost/db?user=USER&password=PASS"\
 -v -l example2-large
\end{lstlisting}


\subsubsection{Example 3}
Total indexing time: {\bf 14750s (4h 4m 50s)} \\
Optimizing time: {\bf 2260s (37m 40s)} \\
Index size: {\bf 22GB} \\

\noindent LuSql command:  \\
Like Example 2, with  {\tt -l example3-large} and appending:
\begin{lstlisting}[backgroundcolor=\color{grey},language=Bash]
-Q "id|select string as keyword from Keyword,\
 ArticleKeywordJoin where ArticleKeywordJoin.articleId = @\
 and ArticleKeywordJoin.keywordId = Keyword.id"\
-Q "id|select lastName, firstName from Author,\
 ArticleAuthorJoin where ArticleAuthorJoin.articleId = @\
 and ArticleAuthorJoin.authorId = Author.id"\
-Q "id|select referencedArticleId as citedId\
 from Reference where Reference.referencingArticleId = @"
\end{lstlisting}

\subsubsection{Example 4}
{\bf NB}: There was not sufficient disk space on the indexing machine to
index. 
Instead, the indexing was done on the database machine.
As both the indexing and database were running in the same machine, it is
almost certain that the indexing time would be significantly less than for the
previous examples, so making comparisons between these results and results
from Examples 2 and 3 would not be appropriate. \\

\noindent Total indexing time: {\bf 49593 s (13h 46m 33s)} \\
Optimizing time: {\bf 9623s (2h 40m 23s)} \\
Index size: {\bf 86GB GB}

\noindent LuSql command:  \\
\noindent Like Example 3 with {\tt -l example4-large}, and appending
\begin{lstlisting}[backgroundcolor=\color{grey},language=Bash]
 -f ca.nrc.cisti.lusql.example.FileFullTextFilter
\end{lstlisting}








