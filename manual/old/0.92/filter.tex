\label{filter}
LuSql offers a plugin architecture which allows the user to define a
class which has access to the {\tt Doc} after it is created but
before it is converted into a Lucene {\tt Document} and this is indexed. 
This allows arbitrary modification of the {\tt Doc} object.


\subsection{Interface {\tt DocFilter}}
A class implementing {\tt DocFilter} only needs to implement the
two methods shown below.
\begin{lstlisting}[backgroundcolor=\color{grey},language=Java]
public interface DocFilter 
{
    public Doc filter(Doc doc); 
    public void onDone();
    public void setProperties(Properties filterProperties);
}
\end{lstlisting}

\subsubsection{Class {\tt Doc}}
\label{doc}
{\tt Doc} is an interface that is LuSql's internal abstract
representation of a document, that exposes a number of methods to allow manipulation:
\begin{lstlisting}[backgroundcolor=\color{grey},language=Java]
public interface Doc 
{
   public void addField(final String name, String value, LuceneFieldParameters lfp);
   public LuceneFieldParameters getFieldParameters(String field);
   public void addFieldParameters(String[] fieldNames, LuceneFieldParameters[] paras);
   public void addField(final String name, String value);
   public void addFieldParameter(String field, LuceneFieldParameters paras);
   public void populate(ResultSet rs, String[] fieldNames) throws SQLException;
   public boolean containsField(String key);
   public void removeField(String name);
   public Iterator<String> getFieldNames();
   public List<String> getFieldValues(String key);
   public void clear();
}
\end{lstlisting}

\subsubsection{Class {\tt LuceneFieldParameters}}
The LuSql class {\tt LuceneFieldParameters} used above is a container class for the
three Lucene {\tt Field} parameters and has the following constructor:
\begin{lstlisting}[backgroundcolor=\color{grey},language=Java]
    public LuceneFieldParameters(Field.Index newIndex,
				 Field.Store newStore,
				 Field.TermVector newTermVector) 
\end{lstlisting}




\subsection{Interface {\tt DBDocFilter}}
Sometimes the class implementing {\tt DocFilter} needs to access
the SQL database being used.
{\tt DBDocFilter} is an interface that extends  {\tt
  DocFilter} and offers access to the {\tt DataSource} object so
it can be used by the filter.
The {\tt DBDocFilterImp} class implements this interface, so
developers can just extend this class to get this functionality.

\begin{lstlisting}[backgroundcolor=\color{grey},language=Java]
public interface DBDocFilter 
    extends DocFilter 
{
    public void setDataSource(DataSource ds);
    public DataSource getDataSource();
}
\end{lstlisting}

See Appendix \ref{filterSource} for an example implementation.












