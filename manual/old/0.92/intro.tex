
LuSql\footnote{LuSql: {\em loo'-skwill}} is an easy to use, high-performance
command-line Java application originally designed to allow users to create
Lucene indexes from the results of a SQL query of a SQL database, but now
extended to be a ZZZZ, allowing data migration from a number of source and
destination platforms using a plug-in architecture.

While LuSql does support JDBC --> Lucene migration directly, it also comes
with a number of drivers for other systems.
These include:
sources:
JDBC, BDB, Java serialized objects
sinks:
JDBC, BDB, XML, text, Lucene, Minion, Solr (via SolrJ), Serialized



\subsection{Why LuSql?}
Many users have their data in a DBMS and some have the need to move
this data to Lucene.
For some, they find that their needs for full-text indexing are not well-met
by their DBMS. 
For others, they need a web facing searchable cache of data stored in their
DBMS.  

Whatever the specific use case, moving this data to Lucene is an effective and
fairly common solution.  
But how to do this?
On the Lucene user mailing
list\footnote{\url{http://mail-archives.apache.org/mod_mbox/lucene-java-user}},
periodically someone asks if there are simple tools for doing this sort of
data transfer:
\begin{lstlisting}[backgroundcolor=\color{grey}]
 To: java-user@lucene.apache.org
 From: "???" <xx.xx.@xxxx.com>
 Subject: Lucene Indexing DB records?
 Date: Fri, 22 Aug 2008 09:26:54 GMT
  
 Guess I don't quite understand why there are so few posts 
 about Lucene indexing DB records. Searched Markmail, but 
 most of the Lucene+DB posts have to do with lucene index 
 management. 
  
 The only thing I found so far is the following, if you 
 have a minute or two:
 http://kalanir.blogspot.com/2008/06/indexing-database-
using-apache-lucene.html
  
 Any comment on this? Efficiency? No need to maintain any 
 intermediate data?
  
 Thanks much,
 -Xxx
\end{lstlisting}


There {\em are} a number of tools which support this sort kind of data transfer
from a DBMS to Lucene, either directly or as part of a larger framework: 
\begin{mlist}
\item SOLR\footnote{\url{http://lucene.apache.org/solr/}}
\item DBSight\footnote{
        \url{http://www.dbsight.net/}
        }
\item Lucene4DB.Net\footnote{
        \url{http://www.netomatix.com/Products/DocumentManagement/Lucene4DB.aspx}
        }
\item Hibernate Search\footnote{
        \url{http://search.hibernate.org/}
        }
\item Compass\footnote{\url{http://www.compass-project.org/}}
\end{mlist}



Without getting into an in-depth analysis of each of these tools, they all suffer
from one or more of the following issues:
\begin{mlist}
\item Overly complicated for non-Lucene / non-Java / non-XML /
  non-framework users
\item Poor performance
\item Not Open Source Software (OSS)
\end{mlist}


LuSql is a simple, high performance, OSS DBMS to Lucene indexing
application that has a very low barrier for use.
Users need the following to use LuSql:
\begin{mlist}
\item Knowledge of SQL
\item Knowledge of their database and tables
\item Ability to set the Java CLASSPATH in a command-line shell
\item Ability to run a command line application
\end{mlist}

At the same time, LuSql offers rich features that make it useful for
some very complex use cases.




\subsection{Features}
LuSql offers control over which
{\tt
    Analyzer}\footnote{\url{http://lucene.apache.org/java/2_3_2/api/org/apache/lucene/analysis/Analyzer.html} 
  }
is applied to the 
{\tt Field}s\footnote{Right now only a single Analyzer is applied to all
  fields. See Section \ref{todo} for more information.} and how/if a
{\tt Field} is stored, indexed, tokenized, compressed, term vectored, etc.

\noindent The SQL query can range from a simple query on a single table: 


\begin{lstlisting}[backgroundcolor=\color{grey}]
  select * from Person;
\end{lstlisting}

\noindent 
to a complex query involving multiple
tables and joins:
%\begin{verbatim}
\begin{lstlisting}[backgroundcolor=\color{grey}]
select Article.id, Article.title, Article.abstract,
 Publisher.name as pub, Journal.title as jo, Journal.issn,
 Volume.number as vol, Volume.coverYear as year, 
 Issue.number as iss from Publisher, Journal, Volume, 
 Issue, Article where Publisher.id =Journal.publisherId and 
 Journal.id = Volume.journalId and Volume.id = Issue.volumeId 
 and Issue.id = Article.issueId;
\end{lstlisting}
%\end{verbatim}

\noindent For extracting additional DBMS information that is difficult or not possible
through a single SQL query, LuSql supports arbitrary per document subqueries.
These subqueries use a field from the original query as a key into other
fields. 

LuSql also offers a plugin architecture for out-of-band
operations on each document {\em after} it is created from the SQL
record but {\em before} it is inserted into the Lucene index. 
An example use case could be where a field in the SQL record is the full path
to a file in the filesystem that contains full-text that the user wants
indexed in a {\tt Field} in addition to the SQL fields already in the document.
 Using the plugin architecture, a Java class can be written and used by LuSql
 to read the file defined by this field and index the contents.



LuSql also offers control over the following:
\begin{mlist}
\item The manner in which specific {\tt Field}s are stored, i.e. index, store,
  and term vector parameters. 
\item Number of documents to add to index from query {\tt ResultSet}
\item Append to or create new Lucene index
\item Stop word list file
\item JDBC URL
\item JDBC driver class\footnote{The default JDBC driver is the MySQL
    driver, {\tt com.mysql.jdbc.Driver}}
\item JDBC {\tt ResultSet} fetch size
\item Lucene RAM buffer size 
\item Number of worker threads; worker queue size
\end{mlist}

\subsubsection{What LuSql is not}
Some of the tools mentioned above offer various features, like automatically
building a web search interface for the created Lucene index.
LuSql only builds a Lucene index.
It is up to the user to decide how they are going to create a user interface
to search the created Lucene index. 
